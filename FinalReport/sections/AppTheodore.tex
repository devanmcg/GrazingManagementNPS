\section{Theodore Roosevelt National Park}

\subsection{Current Grassland Management Goals and Programs}

\subsubsection{Grassland Management Goals}

\paragraph{Lightly utilized prairie}
Grasslands at THRO are lightly stocked. 
This allows the protected lands within the park to contrast the heavily grazed lands surrounding the three units of the park. 
The bison herds at THRO are culled to maintain this light utilization and horse bands are being assessed for contraceptive control to inhibit the growing population. 
The landscape at THRO is inherently heterogeneous. 
A patchy disturbance regime needs to interact positively with this diverse landscape to maintain forage banks for herbivores in the unit. 
Intense disturbance is needed rather than an even stand of lightly utilized prairie.

\paragraph{Conservation and cultural wildlife management} 
Population genetics of wildlife are studied at THRO. 
Genetic testing of bison informs which animals to cull in annual roundups to maintain an even spread across age and sex. 
This aligns with the ``conservation herd'' parameters the MWR sets for bison. 
Genetics are also used in feral horse management. 
The small genetic pool of the band has led to non-viable offspring. 
Feral horses and longhorn cattle are managed as cultural exhibits, although these non-natives species still require significant time on the part of natural resources staff.

\subsubsection{Grassland Management Programs}

\paragraph{Wildlife Management} 
Feral horses, bison, longhorn cattle, and several other herbivores graze in the park.
 Feral horses and longhorn cattle are managed as an interpretive exhibit while bison are managed as a conservation herd. 
Bison are the focus as they are a native grazer, reintroduced in 1956. 
The park maintains 200-400 animals in the south unit of the park and 100-300 animals in the north unit. 
Feral horses are maintained for the cultural aspect and visitor experience. 
Both bison and feral horse genetics are studied extensively to optimize population vitality. 
Finally, longhorn cattle are maintained as an interpretive exhibit in the Elkhorn Ranch Unit to recreate Theodore Roosevelt's ranch. 
Elk were also reintroduced to the south unit in 1985 with a major cull occurring 2010-2013.

\paragraph{Invasive Species} 
In 2018, the park began treating invasive species after a period (2012-2017) without treatment of any kind. 
This gap allowed leafy spurge, Kentucky bluegrass, and Canada thistle to increase (Folluo, 2017). 
Riparian areas are especially invaded with smooth brome and Kentucky bluegrass. 
Priority is placed on the management of these species so that the prairie is not taken over by non-native species.

\textbf{Management Plans and Data Available}

\emph{Management Plans}

Plans at THRO cover a diverse range of resources. 
The Fire Management Plan, written in 2008, although present in our time parameters, could be updated. 
Management goals in the unit have changed. 
Invasive species are increasing, and climactic conditions are changing at an accelerated rate. 
A complete list of published management plans for THRO is shown in Table~\ref{tab:THROmandocs}. 
The vegetation management strategy from 2012 gives a complete background on vegetation management done in the park and previous research completed pertaining to vegetation.

\begin{table}[h]
	\centering
\caption[THRO management documents]
	{Management documents for THRO published from 2008-2018}
\label{tab:THROmandocs}
\begin{tabular}{lc}
	\toprule
	Title & Year\tabularnewline
	\midrule
	Fire Management Plan & 2008 \tabularnewline
	Elk Management Plan  & 2010 \tabularnewline
	Long Range Interpretive Plan & 2011 \tabularnewline
	Vegetation Management Strategy & 2012 \tabularnewline
	Natural Resource Condition Assessment & 2014\tabularnewline
	Foundation Document & 2014 \tabularnewline
	\bottomrule
\end{tabular}
\end{table}

\subsubsection{Data Available}

In 2012, THRO lost their park botanist and efforts related to vegetation ceased for some time. 
I\&M still collected data, but park staff communicated that I\&M data are not the most useful due to the variability present on the THRO landscape. 
Selected data can be seen in Table~\ref{tab:THROdata}. 
Vegetation data collected prior to 2008 can be found in the vegetation management strategy. 
Table~\ref{tab:THROdata} shows the vegetation data gap from 2012-2017. 
Wildlife data is up to date and has recently been the focus of collection and management efforts. 
This lack of time on vegetation is the impetus for vegetation data collection
starting again in 2018.

%\storeareas\normalsetting
\KOMAoption{paper}{landscape}
\areaset{1.5\textwidth}{.6\textheight}
\recalctypearea
\pagestyle{plain}
\setlength\LTcapwidth{1.5\textwidth} 
\setlength\LTleft{0pt}           
\begin{longtable}[l]{@{}p{5cm}p{2cm}p{3cm}p{4cm}p{3cm}p{4cm}p{3cm}@{}}
	\caption[THRO data]
	{Selected data collected in THRO, 2008-2018.} 
	\label{tab:THROdata} \\
	\toprule
	Data title & Data type & Spatial extent & Frequency & Duration & Collecting agency & Format located \tabularnewline
	\midrule
	\endfirsthead 
	\caption* {\textbf{Table \ref{tab:THROdata}}, \emph{continued.}} \\
\toprule
Title of Data & Type of Data & Spatial Extent & Frequency of Collection & Duration of Collection & Agency Collecting Data & Format Located \tabularnewline
\midrule
\endhead
Oil and gas impacts on air quality in federal lands in the Bakken
region: an overview of the Bakken Air Quality Study and first results &
Air & Bakken surrounding region & 48 hr integrated samples, 6 days per
week- 15 February- 6 April & began in 2013 & NPS - Air Resources
Division & unit network\tabularnewline
Pre- Burn Pedestrian Survey of the Halliday Wells Burn Unit at THRO &
Fire & one burn unit & once to assess archeological resources in a
planned burn area & April 20, 2015- May 21, 2015 & Midwest Archeological
Center & unit network\tabularnewline
Acoustic Monitoring Report & Sound & parkwide & 30 day monitoring
periods- 831 sound level meters & 2012-2015 & NPS & unit
network\tabularnewline
THRO Vegetation Monitoring -- 2017 Summary Report & Vegetation & all
three units of the park. Leafy Spurge (20 plots S, 3 N, 5 ER), Canada
Thistle (13 N, 1 ER) & plots each summer & 2008-2012 then lapsed and
began again in 2017 & NPS (Jenna Folluo, RM Volunteer) & unit
network\tabularnewline
Dendroclimatic Potential of Plains Cottonwood from the Northern Great
Plains, USA & Vegetation & north unit of THRO & cored random tree once &
July-October, 2010 & USGS, University of Arizona, University of Arkansas
& unit network\tabularnewline
Plant Community Composition and Structure Monitoring for THRO &
Vegetation & parkwide & once per summer & 2011-current & NGP I\&M &
IRMA\tabularnewline
Exotic Plant Management Team 2008 Annual Report & Vegetation & parkwide
& annual report written each year for work accomplished during the
summer season & EPMT established in 2000 & NGP EPMT & unit
network\tabularnewline
Summary of observations on dieback of ash trees THRO (south unit) &
Vegetation & three sites in the south unit of THRO & once & Aug-10 &
NDSU & unit network\tabularnewline
Herbicide Application Report & Vegetation & parkwide & ~ & ~ & NPS &
paper copy at unit\tabularnewline
Canada Thistle and Leafy Spurge plot data & Vegetation & parkwide & once
per season & 2008-2012 & NPS & unit network\tabularnewline
North Dakota Plant Species of Concern & Vegetation & state wide & ~ &
published 2010 & ND Natural Heritage Inventory & unit
network\tabularnewline
Leafy Spurge Density Charts & Vegetation & parkwide & once per season &
1983-2011 & NPS & unit network\tabularnewline
Leafy Spurge Status Report & Vegetation & parkwide & different areas
treated each year & data as far back as 1970, report published in 2009 &
NPS & unit network\tabularnewline
Plant Community Descriptions of THRO & Vegetation & parkwide & ~ &
describing characteristic landscapes in THRO & NPS & unit
network\tabularnewline
Canada Thistle Status Report & Vegetation & parkwide & annual report
written each year for work accomplished during the summer season &
1997-2009 & NPS & unit network\tabularnewline
Brine Contamination to Aquatic Resources from Oil and Gas Development in
the Willison Basin, US & Water & three sites across the basin & one time
in 2008 & 2008 & USGS & unit network\tabularnewline
Aquatic Resources along the Little Missouri River near THRO & Water &
four sites along the Little Missouri River & once & Aug 30 2011- Sept 1
2011 & University of Wyoming & unit network\tabularnewline
Annual Wildlife Report & Wildlife & parkwide & each year surveys of
different wildlife populations & 2008-2011 & NPS & unit
network\tabularnewline
Elk Surveys & Wildlife & parkwide & one survey each year & 1985-2000,
2010-2012 & NPS & unit network\tabularnewline
Gypsy Moth Trap Record & Wildlife & ~ & ~ & 2017 & ~ & ~\tabularnewline
Pronghorn Migration and Resource Selection & Wildlife & southwest ND &
tracking fall migrations & 2004-2008 & University of Missouri- Columbia
& unit network\tabularnewline
Detections of \emph{Yersina pestis} East of the Known Distribution of
Active Plague in the United States & Wildlife & five NPS units: THRO,
WICA, DETO, SCBL, BADL & once per year in different months & 2009-2011 &
University of South Dakota & unit network\tabularnewline
The prairie dog: a century of confusion and conflict in park management
& Wildlife & NPS units with prairie dogs & assessing data for one time
report & written in 2009 assessing data from history-2009 & NPS & unit
network\tabularnewline
Use of water developments by female elk at THRO & Wildlife & parkwide &
26,081 samples at 7-h intervals & June- September, 2003-2006 & USGS, NPS
& unit network\tabularnewline
Bison Handling Database & Wildlife & parkwide & yearly added to database
& 1985-current & NPS & unit network\tabularnewline
Elk Handling Database & Wildlife & parkwide & yearly added to database &
1985-current & NPS & unit network\tabularnewline
2016 NU Bison Roundup Data & Wildlife & north unit of THRO & each
roundup & this report is from 2016 & NPS & unit network\tabularnewline
Elk Count Data & Wildlife & parkwide & each year surveys of different
wildlife populations & fall 2015 & NPS & unit network\tabularnewline
2010 Bighorn Sheep Capture & Wildlife & north unit of THRO & once &
15-Feb-10 & NPS & unit network\tabularnewline
Description of Bobcat Harvest & Wildlife & state wide & yearly added to
database & 2004-2008 & ND Game and Fish Department & unit
network\tabularnewline
\bottomrule
\end{longtable}
\clearpage
\normalsetting
\pagestyle{fancy} 

\subsection{Disturbance Regime }

\subsubsection{Grazing }

Bison are free to roam in the north unit and south unit of the park. 
Elk and feral horses roam the south unit. 
The goal of herbivory in the park is a lightly utilized landscape in comparison to the surrounding heavily grazed lands. 
Bison are stocked at a conservative rate and forage quality seems to be good (Licht, 2016). 
Some areas of the park are heavily grazed. 
A study is ongoing to determine locations of bison in the park using GPS collars. 
This will help in developing forage use management. 
Grazers can be drawn away from an area through a coupled disturbance regime. 
They can also be drawn to an area that has been recently burned to feed on nutritive new growth.

\subsubsection{Fire }

Fire is another natural disturbance on the landscape at THRO. 
The unit is inherently heterogeneous. 
This complicates goal setting for ecosystem processes. 
Imposed heterogeneity must align with the inherent heterogeneity to create beneficial patches on the landscape. 
Fire has been absent on the landscape at THRO recently. 
Nine fires have occurred in the last ten years with four of those occurring in 2009. 
An 8,000-acre fire occurred in May of 2018 but was the first since 2011. 
The fire return interval for a patch is longer than other units in this study, but fire should occur in some form each year to ensure nutritive new growth for herbivores the following year. 
It would also aid in the fight against woody encroachment and other invasive species prevalent in the unit.

\subsection{Data gaps and suggested research}

\subsubsection{Data Gaps}

The main data gap in the last decade is vegetation related. 
Staff time and money has been focused on wildlife research. 
Vegetation surveys and invasive species treatment was highly documented, but then ceased in 2012. 
The park is beginning to reassess vegetative resources, but the status of the prairie after this management gap is needed. 
This includes assessment of I\&M documented community composition data as well as determining the status of various ecological sites within the park that the I\&M data may not capture. 
Once resource status is quantified, specific data collection can begin to assess beneficial disturbance to the landscape. 
The North Dakota Badlands are highly erosional landform. 
Understanding the erosion rates in highly grazed areas will enlighten the areas that cannot handle as much grazing disturbance.

\subsubsection{Suggested Research}

To develop a disturbance regime in the prescribed manner, there are several areas the park could investigate. 
The first is understanding the status of vegetative resources in the unit. 
Second, collecting productivity data for varying ecosystem types in the park. 
Ecological site descriptions (ESD) are available for the unit. 
Interview responses state the ecosystem is so highly variable that ESDs are not the best tool to use in management prescriptions. 
Site- specific data is necessary especially on an extremely diverse landscape to create disturbance patches. 
Lack of fire in recent years has also allowed woody encroachment to expand in the park. 
Intense fire behavior is necessary to combat woody encroachment once individuals have established. 
Altering season of fire can create more intense fires. 
Understanding how the landscape at THRO responds to altered fire season would benefit resource managers in establishing a disturbance regime to achieve invasive management objectives.

\begin{itemize}
\item Establish cages to determine productivity of the range in various areas of the park
\item Impact of grazers on erosion rates of riparian areas 
 \item Highly grazed areas and effects on that landscape pertaining to high grazing pressure
\item Invasive species response to season of burn
\end{itemize}

\subsection{Management Recommendations}

\subsubsection{Maintain prairie through coupled disturbance regime}

The most significant issue at THRO is invasive species management coupled with the necessary focus of time and money on wildlife species.
Better managing the processes that the prairie depends on will benefit both wildlife and native species competitive ability. 
An important natural process in the THRO landscape is a disturbance regime. 
Coupling disturbances to create a regime will impose heterogeneity on the already diverse landscape at THRO. 
These imposed disturbances can be used to target specific areas for invasive species management. 
There can also be a benefit of burning in different seasons of the year or at different intensities to reduce exotic cover. 
Recently, a prescribed burn was conducted with the goal of lessening woody encroachment. 
To do this, fire intensity needs to be high. 
The fire was 8,000 acres and was much more focused on extent than on intensity of the patch disturbance. 
A smaller patch of disturbance can then concentrate a secondary disturbance such as grazing. 
With uniform light grazing, the park is not creating enough heterogeneity for certain wildlife species. 
Concentrating grazers in a small area post fire creates intensive disturbance for a season. 
This benefits the landscape by creating patches of disturbance while still maintaining other areas of the unit in mid to late seral stages in case of detrimental weather conditions.

