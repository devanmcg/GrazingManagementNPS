\hypertarget{discussion}{%
\section{Discussion }\label{discussion}}

\hypertarget{conceptual-justification}{%
\subsection{Conceptual Justification}\label{conceptual-justification}}

The research team's recommendations focus on the re-establishment of natural disturbance regimes. 
This entails an active management program in which disturbances, such as fire and grazing, are used to create diverse patches on the landscape.

There must be diversity in areas of disturbance, but there must also be diversity in those areas from year to year.
By creating a mosaic on the landscape of different patches, the ecosystem as a whole will be better adapted to evolve, change, and grow with varying threats to the Great Plains. 
As described above, patch burn grazing is one way to create heterogeneity. 
Burning different areas each year creates burned and unburned areas which offer diverse benefits to the ecosystem. 
Newly burned areas are beneficial for forage quality and wildlife habitat.
Unburned areas maintain grass availability in case of drought.

Management techniques to achieve heterogeneity are briefly discussed in both fire management plans and native grazer management plans in some units.

\hypertarget{key-uncertainties-and-data-gaps-1}{%
\subsection{Key Uncertainties and Data Gaps
}\label{key-uncertainties-and-data-gaps-1}}

According to Driscoll et al, there are three areas of required knowledge when looking at ecologically sustainable management and in particular in terms of fire management which is a key management practice of grassland ecosystem management \citep{driscoll2010}. 
The three areas that must be looked into are:

\begin{enumerate}
\item Species response to fire regimes
\item Knowledge of how spatial and temporal characteristics of a fire affect biota
\item How fire regimes interact with other ecological processes
\end{enumerate}

According to a manager survey, most parks in the Great Plains are managed from a species perspective. 
For this reason, there is lots of data collected both at most unit levels as well as at a region level within the units.
The first point made by Driscoll is to better understand the system by considering all species and their response to a disturbance. 
This is incorporating each species into a larger framework of management.

\textbf{Species response to fire regimes.} Things that should be looked at are species dependence on specific habitat resources, spatial distribution of fire and how this influences the availability of limiting resources, development of functional groups (also plays into resilience of a SYSTEM vs resilience of a SPECIES), thresholds of fire behavior, and testing of the predictions. 
Particularly in the face of maintaining ecosystem resilience with changing climactic factors in the region, it will be important to understand how different species respond to fire regimes across their range.
 It is also very important to understand how fire regimes affect the establishment and spatial arrangement of different species. 
One example of this is understanding how invasive species emerge and establish post fire and if spatial variability of fire can influence this cycle to minimize the establishment of new populations of invasive species.

\textbf{Knowledge of spatial and temporal effects of fire}. There are three key pieces to define when establishing a patch burn grazing system on a landscape. 
These include: the stocking rate, the fire return interval and the season of burning. 
This is where further research on the beneficial nature of a variety of different combinations would be beneficial.
Shorter fire return intervals are beneficial for wetter areas while longer fire return intervals benefit drier climates. 
The importance of site specific data cannot be overstated. 
There have been few studies conducted in the northern Great Plains but those that have been conducted have been beneficial in showing that fire does not negatively impact this ecosystem \citep{vermeire2011}. 
The fact that this study displayed the importance of fire, this supports the recommendation of providing resources to reinvigorate this disturbance regime.

\textbf{Interaction of fire regimes and other processes}. Fire is overall, the most intense form of disturbance on grasslands and also the piece that needs the most direct and constant management while it is affecting the landscape. 
Once the season of fire, the spatial extent of fire and the intensity of fire is determined for the most positive effects on the landscape in question, then you can begin to add to this disturbance to exacerbate benefits to the ecosystem. 
We advise looking at how grazing processes interact with fire. 
This takes an understanding of how large your patch sizes created by fire will be and then how many animals those patches can stand. 
Based on how many animals your landscape contains may also cause you to revise the fire regime you have established. 
The interaction will take constant adjustment to create a regime to achieve unit specific goals. 
Unit specific data and research studies to determine the capacity of your landscape for disturbance will be intensive at the outset, but once initial data has been collected to establish a regime, the natural processes should carry out ecosystem functioning and eventually require less constant human control.

Looking at both sides of interactions, the disturbance regime affects the landscape, but factors of the landscape or factors created on the landscape also impact the regime. 
For example, grazing management and invasive species can alter fuel load and constrain burn season.
Understanding the effects that management practices create for the disturbance regime is always something to be cognizant of.

\hypertarget{implications-for-management-1}{%
\section{Implications for Management} \label{implications-for-management-1}}

Our recommendations include focusing prescribed fire and grazing on pre-determined patches within NPS units.
 We also recommend focusing on a system perspective and creating management plans that incorporate several aspects of the ecosystem rather than writing one plan per species. 
Thinking in this systems perspective will slowly change the goals of NPS management from species vitality to ecosystem processes vitality. 
If we want to maintain ecosystems as they are and ``unimpaired for future generations'', the processes that make grasslands what they are should demand the most attention. 
Functional groups of species can perform similar functions in an ecosystem. 
As long as a required function is performed, less pressure needs to be placed on what specific species is performing that function. 
By taking this mind frame, less pressure will be placed on budgets and personnel to maintain unrealistic expectations for grassland ecosystems into the future.

Following along with the three pieces of information previously mentioned, we specifically recommend, region-wide, to ``continue to research the response of specific target species response to fire and specifically prescribed fire in different seasons'' \citep{driscoll2010}.

 Specific studies to focus on and data to collect would be

\begin{itemize}
\item  Spatial and Temporal Characteristics of Fire and How that Affects Biota
\item  How Fire Affects Ecosystem Processes
\end{itemize}

Disturbance regimes are essential processes for native grasslands of the Midwest region. 
By acknowledging this fact and focusing management on the establishment of this disturbance regime could ultimately lead to less management inputs in the year. 
Instilling a disturbance regime creates native heterogeneity across the protected landscape.
Heterogeneity implies diversity of the landscape and all things that depend on or makeup that landscape. 
When diversity is present, an ecosystem is best prepared to absorb shocks and pressures \citep{walker2012}. 
Shocks and pressures are inherent parts of managing a protected area in a surrounding landscape of utilitarian focus. 
A resilient landscape will influence the amount of management time and budget required to maintain essential processes. 
When the landscape is thought of as a system, and that system is poised to rebound, it can be more so left to its own devices with the trust that instilling nature regimes will beget natural processes going forward.

In conclusion, we suggest coupling disturbances into a disturbance regime. 
Two native disturbances on grasslands that align and benefit one another are fire and grazing. 
This has been implemented on one grassland unit in the region with native herbivores in a technique called patch-burn grazing. 
This technique is a way to encourage pyric herbivory on the landscape. 
As this has been seen to benefit the landscape at TAPR, we suggest translating this technique to other units. 
In order to translate and set up this process in new locations, plant productivity is required to prescribe the preferred stocking rate and spatial extent of the patches created by patch- burn grazing. 
New data will need to be collected through grassland exclosures and weighing of vegetative material.