The purpose of this project was to contribute to an understanding of the current state of grassland resource management in five units of the National Park Service's Midwest Region. 
Our specific project goals were to

\begin{enumerate}
	\item Identify multi\textendash species management goals and ecological objectives for each Midwest Region National Park Service unit to enhance grassland ecosystem function
	\item 	Recommend changes to existing strategies and/or specific actions to improve achievement of management goals 
	\item Identify key uncertainties and data gaps that might act as be barriers to managing for resilient grassland ecosystems, and prioritize research needs to address high priority data gaps
\end{enumerate}

We used a three-part approach consisting of on-site interviews with NPS management, on-site and online audits of available documents and data for each unit, and a follow-up survey to establish how broad-based interview responses were across all participants. 

In our analysis and discussion we focus on the \emph{grazing ecosystem}, which \citet{frank1998} distinguish from other habitats "by its prominent herbivore-based food web and by the extent to which ecological processes are regulated by dynamics within that food web." 
Our primary finding is that sustainable management of grazing ecosystems in the NPS Midwest Region would benefit from co\"{o}rdinated management of fire and grazing regimes. 

In grazing ecosystems around the world, the fire-grazing interaction creates a unique ecological disturbance referred to as \emph{pyric-herbivory}, which supports and stabilizes a breadth of grassland ecological services and functions. 
Successfully coupling fire and grazing regimes in a management context requires an integrated perspective that is often substantially different from the species-specific focus we found to predominate natural resource management at each of the NPS units studied here. 

In addition to the unit-specific assessments and recommendations given in the Appendices, this report discusses the science behind fire-grazing interactions in grazing ecosystem management, principal considerations for successfully coupling fire and grazing regimes in conservation areas of the North American Great Plains, and addresses potential cultural and logistical barriers to the adoption of necessary management perspectives and practices. 
We also describe steps that can be taken to facilitate the successful coupling of fire and grazing regimes in the NPS Midwest Region. 
Briefly, these steps include: 
\begin{itemize}
	\item Use ecological site descriptions to identify soils-based disturbance response groups as a means to understand patterns of plant community composition and productivity dynamics at a landscape level. 
	\item Add primary productivity measurements to existing plant community sampling protocols, extend the application of those protocols, and incorporate productivity data with disturbance response groups to delineate burn units that maximize the stability of grazing resources within and across years.
	\item Maximize opportunities to use domestic livestock in grazing ecosystem management as a flexible means to track stocking rate with rainfall and primary productivity. 
\end{itemize} 


