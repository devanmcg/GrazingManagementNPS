\hypertarget{review-of-project-objectives-1}{%
	\section{Review of Project Objectives
	}\label{review-of-project-objectives-1}}

Primary project goals:

\begin{enumerate}
	\item Identify multi- species management goals and ecological objectives for each unit that are expected to result in improved grassland ecosystem function
	\item 	Recommend changes to existing strategies and/or specific actions to improve achievement of management goals 
	\item Identify key uncertainties and data gaps which may be barriers to managing for resilient grassland ecosystems and prioritize research needs to address high priority data gaps
\end{enumerate}

Each unit of the National Park Service is responsible to, ``conserve the scenery and the natural and historic objects and the wild life therein and to provide for the enjoyment of the same in such manner and by such means as will leave them unimpaired for the enjoyment of future generations" (Organic Act 1916). 
Whether a postage stamp size unit or a large multi-site park, the managers must interpret this federal directive. 
The five NPS units in this study, Agate Fossil Beds National Monument (AGFO), Badlands National Park (BADL), Tallgrass Prairie National Preserve (TAPR), Theodore Roosevelt National Park (THRO) and Wind Cave National Park (WICA), differ substantially, making their management practices and needs diverse. 
Thus, we worked with each Unit individually to understand their management goals and motivations. 
Identifying goals and ecological objectives at the outset allowed us to determine if current management strategies use the latest science to achieve their goals.


\hypertarget{introduction-and-background-1}{%
	\section{Introduction and Background
	}\label{introduction-and-background-1}}

The goal of this report is to communicate strategies for increasing grassland resilience. Two ways to achieve this are to (1) focus on ecosystem processes, and (2) maintain native disturbance regimes. 
The rationale for these management needs is given in this report. 
We will communicate the importance of these two foci and give evidence of the NPS units' lack of information or plans to work towards these management goals.

The idea that dynamic ecosystems are imperative to grassland resilience is one of the most important scientific acknowledgments in recent history \citep{thomas1996}. 
Current management practices on grasslands in Midwest Region NPS units follow a very species specific approach. 
Unit expert statements and management plans typically focus on single species and how they are affected by the ecosystem.
A systems perspective focuses on interactions and ecosystem function rather than emphasizing vigor of individual species.

Vegetation and wildlife management are species focused. 
In particular, invasive species management depends on the specific invasive species units must manage \citep{bestelmeyer2012}. 
Four units in the Midwest region have bison with plans for a future Wind Cave satellite herd at AGFO. 
All units contain some kind of native grazer, either a managed herd, such as bison or cattle, or migratory herds, such as pronghorn.

Resilient ecosystems require a departure from species perspectives
management, a very common practice in 20th century park planning \citep{lebel2006}. 
Enhancing ecosystem resilience requires focusing on system level
processes, and less of a focus on requiring certain species to be
present. 
Diversity of pressures on protected lands is increasing.
Ecosystems must be able to receive threats and bounce back continuing to produce their characteristic ecosystem services (i.e., carbon sequestration, wildlife habitat, erosion control, viewscapes, etc.) 
Recommendations included at the conclusion of this report will communicate how specific management practices will maintain ecosystem services via Great Plains grassland resilience.

\hypertarget{LitReview}{%
\subsection{Review of the Literature on Grassland Management}\label{LitReview}}

\textbf{Current Issues}. In grasslands across the Midwest there are many issues that warrant attention. 
Chief among these is fire suppression.
The major issue is the removal of a key disturbance and the alteration of the native disturbance regime in the area. 
One result of this is a buildup of fuels, allowing plant species to reach late successional stages. 
Woody encroachment will alter grassland ecosystems and the wildlife that depends on grassland ecosystems \citep{fuhlendorf2012}.
Native plants are adapted to the native fire regime and perform well under cycles of burning. 
Suppression has also removed burned areas as a means to move animals across a landscape. 
Grazers must be fenced in to keep them in the desired grazing space. 
These fences have partitioned off areas of the landscape, inhibiting wildlife movement and adding fence maintenance costs to already strapped NPS unit budgets.

\textbf{Historic Regimes}. 
Most issues and resulting recommendations addressed in this report will center on the idea of historic disturbance regimes. 
This means that the issue we are addressing is the removal or suppression of historic disturbance regimes and will communicate management actions striving to restore or mimic, as best as possible, historic regimes. 
Grasslands are ecologically adapted to thrive under disturbance \cite{samson2004}. 
The species and processes that make up grasslands need disturbance to flourish. For example, several species of plants require the clearing effect of fire in order to continue their life cycle. 
Also, some prairie obligate species have adapted to different stages of their life cycles in varied seral stage of plants \citep{ricketts2016}. 
Looking to the past and how the grassland has functioned for centuries is an indicator for what the protected ecosystem needs going forward.

\hypertarget{importance-of-disturbance-in-grasslands}{%
	\subsubsection{Importance of Disturbance in
		Grasslands}\label{importance-of-disturbance-in-grasslands}}

\hypertarget{grazing-as-a-disturbance}{%
	\paragraph{Grazing as a Disturbance}\label{grazing-as-a-disturbance}}

\textbf{Grazing. }

\textbf{Stomping}. 
The presence of grazers on the landscape is also beneficial for the large impact that their weighty bodies make on the landscape. 
Grazers physically and chemically disturb the landscape through wallowing, walking, and defecating \citep{allred2011}. 
Without this physical impact, micro heterogeneity is absent which is another key piece of grassland ecosystem vitality for diminutive wildlife species \citep{fuhlendorf2017}. 
Stomping can also disproportionately affect non-native vegetation species. 
As grazing is a native disturbance to grasslands, native species have adapted to this pressure/ disturbance.
Less hardy species may be minimized by intense grazing pressure (i.e. Yellow Flag Iris \citep{spaak2016}. 
On top of the removal of grassland species through grazing, the impact that large grazers physically have on the landscape can give a competitive advantage to native species of wildlife and vegetation.

\hypertarget{fire-as-a-disturbance}{%
\paragraph{Fire as a Disturbance}\label{fire-as-a-disturbance}}

\textbf{Fuels Reduction}. 
In the 20th century, fire suppression was the norm across the National Park Service \citep{umbanhowarjr1996, bachelet2000}. 
If fires began naturally, or were started via a human act, they were extinguished as quickly as possible presumably to protect both humans and resources the unit protects. 
Due to continued suppression, fuels built up across the landscape. 
This led to wildfires breaking out across the plains much harder to suppress due to higher fuel loads causing fires to burn more intensely and across a larger extent.
Instilling fire as a consistent process on NPS landscapes will remove dangerous fuel loading. 
Consistently burning patches year to year will allow fuels to build up so that prescribed fires can continue to be carried, but decades will not pass allowing fuels to build to dangerous and difficult to remove late seral stages of grassland.

\textbf{Native Species Competition}. 
Native prairie species have evolved along with disturbance. 
They have many adaptations (emergence, root structure, seed banks, etc.) which allow them to respond to these disturbances in a positive way \citep{hobbs1992, lawes2013, russell2015, midgley2016}. 
Exotic species may not fare as well or in as predictable a manner as the prairie natives.
Altered seasons of burning can be used to target when invasive species are at a critical juncture in their life cycle in order to burn the individuals making it less likely they will successfully complete their life cycle \citep{mcgranahan2012, mcgranahan2013a}.

Woody encroachment also competes with native species for landscape
resources. 
When grasslands are subject to woody encroachment, water and sunlight are claimed by larger woody species \citep{twidwell2013}.
This does not allow critical grasses the means to flourish meaning less forage for grazers. 
This also creates a domino effect in that as grasses are less vigorous, it allows openings for woody species to germinate and establish. 
Also, grazers are naturally drawn to grasses and if there is less grass productivity due to the beginning of woody encroachment, the grasses that are present will be subject to higher grazing pressure making them less vigorous to compete \citep{briggs2005}. 
Fire is necessary in order to stave off the invasion of woody species into a functioning grassland or else a conversion of the ecosystem could occur.

\hypertarget{importance-of-heterogeneity-in-grasslands}{%
	\subsubsection{Importance of Heterogeneity in
		Grasslands}\label{importance-of-heterogeneity-in-grasslands}}

We must see fire and grazing as critical to ecosystem processes rather than just management tools used whenever time and money allow \citep{fuhlendorf2012}. 
The past century has commonly used the ecosystem management idea of ``command-and-control''. 
This stressed the need for managers to alter the landscape to the point where it was predictable to humans \citep{holling1996}. 
In more recent history, scientists have recognized that variability is essential to ecosystem processes \citep{turner1989, wiens1997,larkin2016}
This variety which includes physical characteristics of the landscape can be referred to as heterogeneity of an ecosystem. 
It is also important to instill a ``shifting mosaic'' so that ecosystem processes change over time and are not consistently in one space \citep{fuhlendorf2004}. Creating coupled disturbance regimes of fire and grazing begets diversity which begets resilience of a grassland ecosystem.

\textbf{Diversity of Habitat}.
Wildlife in the Great Plains has adapted to variability in the landscape. 
Some of the more imperiled species in this region have come to this status due to alterations in their habitat. 
An example of this is the greater prairie chicken (GPC). 
The GPC, like many prairie obligate species, requires multiple successional states to successfully reproduce \citep{svedarsky2003}. 
Early succession stage patches for mating and late succession stage patches for nesting. 
Homogeneous landscapes do not satisfy one of these preferred stages thus making it more likely the prairie chicken will not successfully reproduce. 
This is just one example, but there are a
diversity of grassland birds that have proven to benefit from habitat heterogeneity \citep{churchwell2008, coppedge2008, stroppel2009, hovick2014}. 
Benefits include an increase in species richness and density.

\textbf{Forage Quantity and Quantity}. 
Heterogeneity begets variation in grassland successional stage. 
When dealing with a highly variable climate, as is typical across the Great Plains region, having a variety of grass stands may be the difference in the future health of your grassland \citep{mcgranahan2014}. 
By introducing a heterogeneous landscape, patches are created of vegetation that are in different stages of their development. 
Not only is this beneficial for habitat, but it also benefits herbivores' diet. 
Patches increase the resilience of the landscape to a variety of stressors which may impact forage
production \citep{fuhlendorf2017}. 
This could be anything from drought, to development, to unplanned fires. 
All of these have the ability to create a stand replacing disturbance, leaving grazers in your protected area no options for sustenance. 
These situations can be insured against with a patchy landscape.

On top of the availability of forage generally, it has also been proven that the combination of fire and grazing has the ability to increase the level of nitrogen availability \citep{anderson2006}. 
This study proved that the coupled disturbance regime is even more beneficial in terms of nitrogen availability than simply using fire alone to create disturbance of a grassland.

\textbf{Resilience}. 
The inherent nature of heterogeneity means that a landscape demonstrating this would be more able to adapt to variable conditions either on a seasonal basis or over several years. 
For example, if a year of drought hits, the current year new growth may not fare well enough to support a grazing herd of bison. 
A patch of grass that has not been burned may be designated as a forage bank for this particular reason \citep{mcgranahan2013a}. 
Another aspect of resilience comes in the form of wildlife presence due to patchiness of habitat available. 
When there are many different habitat types available for species, it encourages the presence of a diverse population \citep{moranz2012,ricketts2016}. 
The more diverse a
population of wildlife or vegetation is, the more resilient an ecosystem can be in continuing to provide a variety of ecosystem services
\citep{peterson1998,walker2012}. 
This is especially important in the face of a barrage of pressures in the future.

\hypertarget{unit-specific-priorities-and-constraints}{%
\subsubsection{Unit Specific Priorities and
		Constraints}\label{unit-specific-priorities-and-constraints}}

``Fire management and ungulate management are influenced by spatial and political factors at least as much as by scientific factors and choices are partially informed by science but ultimately driven by human values'' \citep{cole2012}. 
For this reason, although we can look to the scientific literature for ways to better increase the functioning of grassland ecosystems, all the suggestions may not take. 
This is why we also answered the call of this project by talking with managers at the units to best understand what priorities they have in management. 
The enabling legislation of units guides the focus of management, budget and staffing needs for park units throughout the country. 
There are also external and internal pressures which guide what tasks are preferred.
Supporting our recommendations with social characteristics of the units will better enable the best scientific suggestions to be implemented within a diverse cross section of park service units.

In this study we used qualitative and quantitative methods to achieve the outlined goals.
 By using both of these methodologies, we could
better build and understanding of management in the units and both what data is physically not present and what the managers feel needs to be more of a focus in management. 
Results are communicated by way of data audits, interviews, and surveys. 
This gives strength to our results presented and makes our recommendations extremely relevant to the NPS units for whom this study was undertaken.

