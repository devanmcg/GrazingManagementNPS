\section{Introduction}\label{sec:intro}

\subsection{Review of Project Objectives}\label{ssec:objectives}

Primary project goals:

\begin{enumerate}
	\item Identify multi\textendash species management goals and ecological objectives for each Midwest Region National Park Service unit that are expected to result in improved grassland ecosystem function
	\item 	Recommend changes to existing strategies and/or specific actions to improve achievement of management goals 
	\item Identify key uncertainties and data gaps that might act as barriers to managing for resilient grassland ecosystems, and prioritize research needs to address high priority data gaps
\end{enumerate}

Across the United States, each unit of the National Park Service (NPS) has a mandate to ``conserve the scenery and the natural and historic objects and the wild life therein and to provide for the enjoyment of the same in such manner and by such means as will leave them unimpaired for the enjoyment of future generations \citep{organicact1916}''. 
Whether a postage stamp size unit or a large multi-site park, the managers must interpret this federal directive. 
We focus on five NPS units in the Midwest Region: Agate Fossil Beds National Monument (AGFO), Badlands National Park (BADL), Tallgrass Prairie National Preserve (TAPR), Theodore Roosevelt National Park (THRO) and Wind Cave National Park (WICA).
The size and specific mandate of each Unit differs substantially, making their management practices and needs diverse. 
Thus, we worked with each Unit individually to understand their management goals and motivations. 
Identifying goals and ecological objectives at the outset allowed us to determine if current management strategies use the latest science to achieve their goals.

The purpose of this report is to communicate strategies that enhance the resilience of grassland resources for NPS units that both rely upon, and are required to conserve, functional native prairie ecosystems. 
We focus on ensuring highly-functional ecosystem processes through the restoration and maintenance of natural ecological disturbance regimes. 
The rationale for this management approach follows in this report, and we highlight areas where units in the region lack of information or plans to work towards these management goals.

\subsection{Grassland ecology and management}\label{ssec:ecol}

Primary drivers of species composition and ecological structure and function in grasslands are fire, herbivory, and climate \citep{blair2014}.
Grasslands are ecologically adapted to thrive under ecological disturbances \citep{samson2004}. 
Considered together, intensity, frequency, spatial extent, and temporal characteristics (e.g., seasonality and duration) of an ecological disturbance comprise the \emph{disturbance regime}. 
Most issues and resulting recommendations addressed in this report focus on the suppression, removal, or alteration of historic disturbance regimes. 
In this report, we describe management actions that seek to restore or mimic, as best as possible, the disturbance regimes that characterized the grasslands of the Northern Great Plains prior to European settlement. 
Here we review the scientific literature that forms our basis or natural or ecologically-analogous disturbance regimes.  

\subsubsection{Fire}

Fire has been an intrinsic component of grasslands and other rangeland ecosystems since these biomes developed and spread around the world. 
Grasslands as we know them today\textemdash including both the vegetation and characteristic animal communities\textemdash emerged after the last Ice Age, and have burned regularly and naturally since. 
Many grass-dominated landscapes owe their existence to fire, especially in the US Great Plains \citep{axelrod1985}, where precipitation is substantial enough to support woody plant species capable of converting these landscapes to shrublands, woodlands, or even forests without regular burning.

As Europeans established colonies and new nations around the world, they radically altered natural fire regimes. 
Settlers on the North American prairies especially saw fire as an existential threat to homesteads and towns made of flammable materials with little capacity for defense against fires. 
It is possible that the fires Europeans encountered were extraordinarily intense due to unnaturally high fuel loads following the extirpation of both bison and indigenous peoples\textemdash which prevented large fires by grazing and intentionally burning, respectively\textemdash prior to settlement; see \citet{courtwright2011}. 
As natural resource management became institutionalized in the 20\textsuperscript{th} century, the perspective that fire was a force of destruction that needed to be stopped to protect valuable natural resources became central to federal management policy, including the National Park Service \citep{umbanhowarjr1996, bachelet2000}. 
If fires began naturally, or were started via a human act, they were extinguished as quickly as possible presumably to protect both humans and resources the unit protects. 
Due to continued suppression, fuels built up across the landscape. 
This led to wildfires breaking out across the plains much harder to suppress due to higher fuel loads causing fires to burn more intensely and across a larger extent.

The view of fire as a destructive force was\textemdash and in many parts of the US, remains\textemdash a tenet of rangeland management aimed at maximizing livestock production. 
The analogies between fire and grazing animals as consumers of vegetation are obvious \citep{bond2005}, and many livestock producers view combustion as a competitor to rumination, literally vaporizing\textemdash and thus wasting\textemdash plant biomass that could be turned into marketable animal products. 
Meanwhile the historical association between fire and loss of life and property persists in the psyche of residents throughout the Great Plains \citep{courtwright2007}, which perpetuates the narrative that fire not only consumes forage but has an overall negative effect \citep[e.g.,][]{wright1982}. 

Early ecologists struggled to reconcile the destructive nature of fire with the persistence of fire-dependent grassland ecosystems in areas that should otherwise support trees \citep{clements1916,transeau1935}. 
The current perspective is that rangelands are dynamic, \emph{non-equilibrium} systems that require periodic disturbances like fire to maintain optimal ecological function \citep{westoby1989}. 
Meanwhile conventional rangeland management aimed at maximizing livestock production has been implicated in the degradation of habitat for many of the species that contribute to grassland biodiversity \citep{fuhlendorf2001}. 
Fire is now widely recognized as critical to balancing the ecology and productivity of rangeland \citep{toombs2010, fuhlendorf2012}.

Rangeland managers in the Great Plains are increasingly adopting fire \citep{twidwell2013}, although today's fire looks different than those of previous centuries. 
Wildfires\textemdash whether the result of a natural or human-caused ignition\textemdash are still suppressed, and managers implement prescribed fires to ensure land burns under conditions that provide as much safety and control over fire behavior and fire effects as possible. 
In most rangelands, the effects of having been burned are no longer apparent approximately two years after a prescribed fire  \citep{limb2016}.  


\subsubsection{Grazing}

Grazing is a general type of herbivory in which consumers non-selectively procure mouthfuls of vegetation, and is contrasted with browsing, in which consumers carefully select individual tissues from individual plants. 
As a result, grazers typically focus on finer plant material such as grasses, other graminoids, and herbaceous flowering plants (forbs), and have wide mouths and tongues to facilitate large bites; browsers can select leaves from woody plants with narrow mouths and nimble tongues. 
Bison, cattle, and sheep are grazers, while deer and goats are browsers. 
(Pronghorn antelope and elk, meanwhile, are versatile herbivores, switching between grazing and browsing depending on season and nutritive demands.)
Evolutionary interactions between grazer morphology, behavior, forage preferences, and plant-soil dynamics have created a global pattern of "grazing ecosystems" as diverse as Yellowstone National Park and the Serengeti \citep{frank1998}. 

The ecological integrity of many grazing ecosystems worldwide has been threatened by the introduction of commercial agricultural grazing regimes. 
Ranchers and pastoralists often manage domestic livestock with high stocking rates and rely on infrastructure and other inputs to increase productivity, frequently with negative impacts to biodiversity and ecosystem function \citep{watkinson2001}. 
In the western United States, livestock grazing has been implicated in the degradation of substantial public land area \citep{fleischner1994}. 
In its emphasis on spatially-even forage utilization in pursuit of maximum livestock production, commercial range management in the Great Plains has failed to support critical habitat and other ecosystem services for rangeland biodiversity \citep{fuhlendorf2001}.
At the same time, herbivory is a defining ecological component of grazing ecosystems and in many grasslands is essential to maintain biodiversity and productivity \citep[e.g.,][]{collins1998, patton2007}. 
With the extirpation of native grazers from most of North America's grasslands\textemdash and many Midwestern NPS units\textemdash a balance must be struck between sustainability and biodiversity conservation \citep{watkinson2001}.

A critical distinction between management and grazer identity must be made when describing and planning a grazing regime. 
While intensive livestock management can effect environmental damage to soil, slopes, and water quality \citep{bilotta2007}, well-managed grazing systems are "a forgotten hero of conservation" \citep{franzluebbers2012}.
In the Great Plains, how grazers are managed tends to matter more for conservation outcomes than whether the species is native (bison) or introduced (cattle) \citep{allred2011a}. 
Cattle grazing, specifically, was identified as having a positive influence on some grassland-obligate birds on public grassland in the Northern Great Plains \citep{ahlering2016}.

\subsubsection{Climate}

Climate zones defined by precipitation, latitude, and elevation divide grasslands of the North American Great Plains into four major biomes\textemdash the eastern tallgrass prairie; western shortgrass prairie, or shortgrass steppe; and two central mixed-grass prairie types, divided into southern mixed-grass and northern mixed-grass. 
With the exception of Tallgrass Prairie National Preserve, in eastern Kansas, the NPS units of the Midwest Region are situated in the northern mixed-grass prairie. 
The western side of the biome is relatively higher elevation than other prairie in the region, which combined with the latitude creates a cool temperate climate characterized by long cold winters, short, potentially hot summers with long days bookended by wide cool growing seasons, and considerable variability in rainfall among years and among seasons with years. 
This climate facilitates perennial grasslands co-dominated by cool-season (C\textsubscript{3}) and warm-season (C\textsubscript{4}) grasses, and a diversity of forbs.

Within local climate zones (sub-types of the broader northern mixed-grass prairie), precipitation is the main driver of short-term primary productivity and a major influence on  plant community composition. 
In fact, despite the importance of fire and grazing on the composition of grassland communities in the Northern Great Plains, precipitation is a moderating influence on disturbance-driven effects on composition \citep{ahlering2016}. 
Primary productivity generally increases with light-moderate grazing, but ultimately tracks with annual precipitation \citep{patton2007}.
Variability in primary productivity and vegetation composition among burned and unburned sites in the northern mixed-grass prairie is negligible compared to the effects of precipitation \citep{whisenant1989, erichsen-arychuk2002}. 

There is considerable uncertainty about how grassland communities in the northern Great Plains will respond to potential alterations to climate regimes under projected global change \citep{chamaille-jammes2010, mcgranahan2018}. 
Shifts in the dynamics of the characteristic co-dominance between C\textsubscript{3} cool-season and C\textsubscript{4} warm-season grasses could be driven by any combination of the following: 

\begin{itemize}
	\item Temperature changes\textemdash{ }Hotter summers should favor the drought-tolerant C\textsubscript{4} photosynthetic pathway, but longer frost-free cool-season growing periods would favor C\textsubscript{3} grasses, especially cool-season invasive species such as Kentucky bluegrass (\emph{Poa pratensis}) and smooth brome (\emph{Bromus inermis}). 
	\item Precipitation changes\textemdash{ }Even among the drought-tolerant C\textsubscript{4} grasses in the region, shifts among C\textsubscript{4} species following sustained deviations from normal precipitation alter forage production and palatability. 
	And even if community composition remains stable, precipitation variability can have substantial impacts on grazers \citep{craine2013a}. 
	\item Atmospheric carbon fertilization\textemdash{ }Atmospheric carbon dioxide concentrations ([CO\textsubscript{2}]) have nearly doubled in the last 250 years. 
	The effects on plant community dynamics are not entirely clear. 
	Generally, the C\textsubscript{3} pathway should be more likely to take advantage of excess carbon in the atmosphere \citep{temme2013}, but there are apparently no overarching differences among C\textsubscript{3} and C\textsubscript{4} grasses, specifically \citep{wand1999}. 
	Changes in soil moisture and water use efficiency are confounding \citep{chamaille-jammes2010}. 
	Other responses to altered [CO\textsubscript{2}] include changes in the nutritional value of plant material \citep{craine2010, taub2008}.
\end{itemize}

\subsection{Ecological Site Descriptions}\label{ssec:ESDs}

Fire, grazing, and climate variability are incorporated into working understandings of non-equilibrium rangeland dynamics through \emph{state-and-transition models} \citep{westoby1989}, adopted widely by the US Natural Resource Conservation Service (see an example in Fig.~\ref{fig:STM}). 

\begin{figure}[!t]
	\center
	\includegraphics[width=1\columnwidth]{loamySTM}
	\caption[State-and-transition model for loamy sites, MLRA 54]{An example of a state-and-transition model from the Ecological Site Description (ESD) for loamy sites in MLRA 54 (NW South Dakota and SW North Dakota). \href{https://esis.sc.egov.usda.gov/Welcome/pgReportLocation.aspx?type=ESD}{\textcolor{BisonGreen}{All NRCS ESD information is available online}}.}
	\label{fig:STM}
\end{figure}

In this system, rangeland landscapes are divided into \emph{ecological sites} defined primarily by soil type (the relative amount of clay, silt, and loam) and landscape position (uplands, bottomlands, or some position along a slope). 
Each ecological site has several potential plant communities; which one is observed depends on management history and environmental factors. 
The potential communities are divided into \emph{states}, or broad condition categories ranging from relatively pristine prairie to  rangeland heavily degraded by invasive species or excessive stocking. 

Within each state are variations of the community that might shift slightly (described by \emph{transitions}) depending on precipitation or time since a non-degrading disturbance like moderate grazing or prescribed fire. 
Large transitions between states occur in response to intense and/or persistent disturbances, and require substantial management intervention to reverse.
In this way managers can predict vegetation responses to disturbance and plan management based on expected productivity under various management and climate scenarios.

\subsection{Grassland resilience}\label{ssec:resilience}

While grasslands are one of the most abundant ecosystems worldwide, in developed countries grasslands are also one of the most imperiled ecosystems \citep{blair2014, newbold2016}.
This is especially the case in the U.S. Great Plains, where 30\textendash 85\% of native tallgrass and mixed-grass prairie has been lost since the mid-1800's, mostly to cultivation \citep{samson1994}; the conversion to row-crop agriculture has continued into the 21\textsuperscript{st} century \citep{wright2013}. 
Most National Park Service units in the Midwest are charged with the conservation of the natural and cultural histories of the North American prairie.
Ensuring an attractive visitor experience in open grassland can be a challenge without mountains, forests, and the exciting geology of Western parks, thus four of the five units in the Midwest Region currently have bison (\emph{Bison bison}) with plans to establish a herd at the fifth (AGFO). 

Overall, current grassland management practices in Midwest Region NPS units follow a species-specific approach reminiscent more of wildlife population management rather than ecosystem ecology. 
The focus on individual species was common in park planning throughout the 20\textsuperscript{th} century \citep{lebel2006}. 
We found both vegetation and wildlife management in Midwest region NPS units to be species oriented. 
In particular, invasive species management depends on the specific invasive species units must manage \citep{bestelmeyer2012}. 

In contrast to classical species-specific wildlife management, an ecosystem perspective considers interactions between communities, the abiotic environment, ecosystem function rather than emphasizing the population status of individual species.
The importance of dynamic ecosystems to grassland resilience is one of the most important scientific acknowledgments in recent history \citep{thomas1996}.
As used here, \emph{resilience} refers to 
\begin{quote}
	The capacity of a system to absorb disturbance and reorganize while undergoing change so as to still retain essentially the same function, structure, identity, and feedbacks \citep{walker2004}. 
\end{quote}

Resilience is important to the NPS if it is to successfully follow its mandate to preserve natural or cultural histories for the education and enjoyment of future generations.
While this might sound like a static mission\textemdash the conservation of a stable state\textemdash it ironically requires substantial flexibility in management to maintain a constant state despite multiple variable environmental factors. 
The diversity of pressures on protected lands is increasing.
Ecosystems must be able to receive threats and bounce back to continue to provide ecosystem services (i.e., carbon sequestration, wildlife habitat, erosion control, viewscapes, etc.). 
Recommendations in this report communicate how specific management practices can enhance the resilience of grassland ecosystems and maintain ecosystem service delivery in the Midwestern NPS units. 

\begin{quote}
Fire management and ungulate management are influenced by spatial and political factors at least as much as by scientific factors and choices are partially informed by science but ultimately driven by human value \citep{cole2012}.
\end{quote}

While this report leans heavily on the published scientific literature to determine how to enhance the function and stability of grazing ecosystems in the NPS Midwest Region, the project has had an inherent human dimensions component from the beginning.
Site visits included not only data audits but semi-structured interviews with park managers at the units to better understand management priorities within existing bounds of social, cultural, and political constraints. 
The enabling legislation of units guides the focus of management, budget, and staffing needs for NPS units throughout the country. 
Prorities and activities are also affected by external and internal forces.
Considering our broad recommendations within the social contexts of specific units will better enable implementation of scientific suggestions.
Thus our approach consisted of both qualitative and quantitative methods to achieve the outlined goals.
These methodologies allowed us to develop unit-specific and region-wide perspectives on current management and management needs. 


