The three goals of this project were to: 
\begin{itemize}
	\item {Identify multi- species management goals and ecological objectives for each park that are expected to result in improved grassland ecosystem function}
	\item {Recommend changes to existing strategies and/or specific actions to improve achievement of management goals}
	\item {Identify key uncertainties and data gaps which may be barriers to managing for resilient grassland ecosystems and prioritize research needs to address high priority data gaps}
\end{itemize}
\subsection*{Current Management Programs}
Each unit of the National Park Service is responsible to, ``conserve the scenery and the natural and historic objects and the wild life therein and to provide for the enjoyment of the same in such manner and by such means as will leave them unimpaired for the enjoyment of future generations." Whether a postage stamp size park or a large multi-site park, the managers within that unit must interpret this federal directive. The five NPS units in this study (AGFO, BADL, TAPR, THRO, and WICA)(from here referred to as ``the park units") all contain vastly different characteristics which make their management practices and needs diverse. For this reason, it was imperative to work with each individual parks to understand what their goals are for management and thus the reason they may make certain management decisions. 
In identifying goals and ecological objectives at the outset of this project, we can see if current park management strategies are using the latest science to achieve their goals. Goals may remain the same, but the scientific rationale for why certain management efforts may be more beneficial than others may change over time as new information becomes available through the scientific research community. For example, in the 1970’s and 1980’s there was a belief and subsequent management actions taken to suppress all fires in this region. It wasn’t until the 1990’s that fire was finally reintroduced to parks in the Midwest region and seen as a beneficial and necessary aspect of ecosystem functioning. 
\subsection*{Recommended Management Alterations}
As scientific understanding evolves, so does the on-the-ground implementation of management in order to achieve desired objectives. There becomes better way to accomplish goals. This happens in every field of science, but particularly in the field of ecology as scientists are constantly discovering new ways that aspects of an ecosystem interact with one another. When these effects are better understood, recommendations more accurately depict how certain actions will affect resources in different ways. The ecology of grassland ecosystems has recently hit a golden era of scienti fic understanding. Researchers in this field are developing technologies and practices that benefit the grass itself but also conserve and benefit the biodiversity of the ecosystem as a whole.
\subsection*{Expected Outcomes Achieved Via Altered Management Programs}
Resilience is one of the most important things to manage for in the current age of accelerating changes of natural processes. The goal of our recommended alterations will all be in the pursuit of resiliency of the grassland ecosystems that the five park service units in this study protect. When resilience is the focus of management, it puts a systems focus on the ecosystem rather than looking at how species respond to specific actions. Resilience is based on diversity and a coupled disturbance regime containing fire and grazing can achieve this diversity in the Great Plains and there have been studies across this region to demonstrate.  