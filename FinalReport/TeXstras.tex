\title{Identifying existing knowledge and necessary data on grassland resources to support resilient grazing ecosystems} % The article title

\newcommand{\ShortTitle}{Grassland resources \& grazing ecosystems}
\let\runtitle\ShortTitle

\author{
	\authorstyle{Betsey York, MSc\textsuperscript{1,2} \\
		Devan Allen McGranahan, PhD\textsuperscript{1}} 
	\newline\newline % Space before institutions
	\textsuperscript{1}
	\institution{Range Science Program, School of Natural Resource Sciences, North Dakota State University}\\ 
	\textsuperscript{2}
	\institution{Oklahoma Department of Conservation}\\ 
}

\date{\today} 

%
% Packages and Settings
%

\usepackage{kantlipsum}

\usepackage{multicol}

\usepackage[svgnames]{xcolor} 
	\definecolor{BisonGreen}{RGB}{0,102,51}
	\definecolor{BisonYellow}{RGB}{255,204,0}

\usepackage{transparent}

\usepackage{longtable, 	lscape}
	\setlength{\LTleft}{0pt}
\usepackage{typearea}
\usepackage{setspace}
%\doublespacing
\usepackage[
colorlinks=true, 
allcolors = {BisonGreen}]{hyperref}
\usepackage{indentfirst}

\setcounter{secnumdepth}{-2} % Remove heading numerals
% TOC stuff 
\usepackage[titletoc, title]{appendix} % How appendix appears

% Fonts & characters
\usepackage[utf8]{inputenc} 
\usepackage[T1]{fontenc}

\renewcommand{\familydefault}{\sfdefault}
%\usepackage{helvet}
%\usepackage{avant}
\usepackage{epigrafica}

% Figures & captions
\usepackage{graphicx} 
\graphicspath{{figures/}}
\setkeys{Gin}
	{width=\linewidth,
	totalheight=\textheight,
	keepaspectratio}

\newcommand*{\plogo}{\includegraphics[width=0.5\textwidth]{NDSURangeScience.pdf}} 

\newcommand*{\R}{\textup{\textmd{\textsf{R{ }}}}}
\newcommand*{\sw}[1]{\texttt{\textrm{#1}}}

\usepackage[english]{babel} % English language hyphenation
\addto{\captionsenglish}
	{\renewcommand{\abstractname}
	{\textcolor{BisonYellow}
	{Summary}}}
\usepackage{microtype} % Better typography
%\usepackage{amsmath,amsfonts,amsthm} % Math packages for equations
\usepackage{chngcntr} % reset counters 
\usepackage[footnotesize,
			labelfont=bf, 
			up]			
		{caption} % Custom captions under/above tables and figures

\usepackage{subcaption}

\usepackage{booktabs} % Horizontal rules in tables
%\usepackage{lastpage} % Used to determine the number of pages in the document (for "Page X of Total")
\usepackage{sectsty} % Enables custom section titles
\allsectionsfont{\usefont{OT1}{phv}{b}{n}} % Change the font of all section commands (Helvetica)

%----------------------------------------------------------------------------------------
%	MARGINS AND SPACING
%----------------------------------------------------------------------------------------
\usepackage{changepage}
\usepackage[ % margins
	top=1.5cm, 
	bottom=1.5cm, 
	left=2cm,
	right=2cm, 
	includehead, 
	includefoot
		]{geometry}

\setlength{\columnsep}{5mm} % Column separation width

\usepackage{enumitem} % Customise lists
\setlist{noitemsep} % Remove spacing between bullet/numbered list elements

%----------------------------------------------------------------------------------------
%	HEADERS AND FOOTERS
%----------------------------------------------------------------------------------------

\usepackage{fancyhdr} % Needed to define custom headers/footers
\pagestyle{fancy} % Enables the custom headers/footers

\renewcommand{\headrulewidth}{0.0pt} % No header rule
\renewcommand{\footrulewidth}{0.4pt} % Thin footer rule

\renewcommand{\sectionmark}[1]{\markboth{#1}{}} % Removes the section number from the header when \leftmark is used

%\nouppercase\leftmark % Add this to one of the lines below if you want a section title in the header/footer

% Headers
\lhead{} % Left header
\chead{\textit{\runtitle}} % Center header - currently printing the article title
\rhead{} % Right header

% Footers
\lfoot{} % Left footer
\cfoot{\footnotesize \thepage\ } % Center footer
\rfoot{} % of \pageref{LastPage}} % Right footer, "Page 1 of 2"

\fancypagestyle{firstpage}{ % Page style for the first page with the title
	\fancyhf{}
	\renewcommand{\footrulewidth}{0pt} % Suppress footer rule
}

\fancypagestyle{lscapefancy}{%
	\fancyhf{}
	\renewcommand{\footrulewidth}{0pt}
}

%----------------------------------------------------------------------------------------
%	TITLE SECTION
%----------------------------------------------------------------------------------------

\newcommand{\authorstyle}[1]{{\large\usefont{OT1}{phv}{b}{n}\color{BisonGreen}#1}} % Authors style (Helvetica)

\newcommand{\institution}[1]{{\footnotesize\usefont{OT1}{phv}{m}{sl}\color{Black}#1}} % Institutions style (Helvetica)

\usepackage{titling} % Allows custom title configuration

\newcommand{\HorRule}{\color{BisonYellow}\rule{\linewidth}{1pt}} % Defines the gold horizontal rule around the title

\pretitle{
	\vspace{-30pt} % Move the entire title section up
	\HorRule\vspace{10pt} % Horizontal rule before the title
	\fontsize{20}{20}\usefont{OT1}{phv}{b}{n}\selectfont % Helvetica
	\color{BisonGreen!120} % Text colour for the title and author(s)
}

\posttitle{%
	\par\vskip 15pt 
	\fontsize{14}{5}
	\color{BisonYellow!150}
	\textit{Status of\textemdash and recommendations for\textemdash data, knowledge, \\
		and management in the 
		NPS Midwest Region}
	\par\vskip 15pt} % Whitespace under the title

\preauthor{} % Anything that will appear before \author is printed

\postauthor{ % Anything that will appear after \author is printed
	\vspace{10pt} % Space before the rule
	\par\HorRule % Horizontal rule after the title
	\vspace{20pt} % Space after the title section
}




%----------------------------------------------------------------------------------------
%	ABSTRACT
%----------------------------------------------------------------------------------------

\usepackage{lettrine} % Package to accentuate the first letter of the text (lettrine)
\usepackage{fix-cm}	% Fixes the height of the lettrine

\newcommand{\initial}[1]{ % Defines the command and style for the lettrine
	\lettrine[lines=3,findent=4pt,nindent=0pt]{% Lettrine takes up 3 lines, the text to the right of it is indented 4pt and further indenting of lines 2+ is stopped
		\color{BisonYellow}% Lettrine colour
		{#1}% The letter
	}{}%
}

\usepackage{xstring} % Required for string manipulation

\newcommand{\lettrineabstract}[1]{
	\StrLeft{#1}{1}[\firstletter] % Capture the first letter of the abstract for the lettrine
	\initial{\firstletter}\textbf{\StrGobbleLeft{#1}{1}} % Print the abstract with the first letter as a lettrine and the rest in bold
} 

\renewenvironment{abstract}{\color{BisonGreen}}

%
% Boxes
%

\usepackage[framemethod=tikz]{mdframed}

\mdfdefinestyle{boxstyle}{%
	linecolor=BisonGreen,
	outerlinewidth=2pt,%
	frametitlerule=true,
	frametitlefont=\centering\sffamily\bfseries\color{BisonGreen},%
	frametitlerulewidth=1pt,
	frametitlerulecolor=BisonGreen,%
	frametitlebackgroundcolor=BisonYellow!40,
	backgroundcolor=BisonGreen!10,
	fontcolor=black,
	innertopmargin=\topskip,
	roundcorner=10pt
}
\newmdenv[style=boxstyle]{exa}
\newenvironment{mybox}[1]
{\begin{exa}[frametitle=#1]}
	{\end{exa}}

\sectionfont{\color{BisonGreen}}
\subsectionfont{\color{BisonGreen!80}}
\subsubsectionfont{\color{BisonGreen!60}}

%----------------------------------------------------------------------------------------
%	BIBLIOGRAPHY
%----------------------------------------------------------------------------------------

%
% %  B I B L I O G R A P H Y 
%
\usepackage[citestyle=authoryear-comp,
sorting=nyt, % name, year, title
%bibstyle=science, % Super-abbreviated like from Science
%bibstyle=authoryear,
bibstyle = nature, 
giveninits=true,
mincitenames=1,
maxcitenames=2,
maxbibnames=2,
minbibnames=1,
hyperref=true,
%editor=false,
uniquename=false,
uniquelist=false,
natbib=true, 
backend=biber]{biblatex}
% Customize bibliograhpy when bibstyle = authoryear: 
\DeclareFieldFormat[article]{title}{#1}  % remove quotes from title
\DeclareFieldFormat[article]{pages}{#1}  % remove Pp. from pages

\renewcommand*{\nameyeardelim}{\addspace} % remove comma after in-text citations
\renewbibmacro{in:}{%
	\ifentrytype{article}{}{\printtext{\bibstring{in}\intitlepunct}}}

% Hide certain fields from biber
% ISSN
\AtEveryBibitem{\clearfield{issn}}
\AtEveryCitekey{\clearfield{issn}}
% ISBN
\AtEveryBibitem{\clearfield{isbn}}
\AtEveryCitekey{\clearfield{isbn}}
% URL
\AtEveryBibitem{\clearfield{url}}
\AtEveryCitekey{\clearfield{url}}
% DOI
\AtEveryBibitem{\clearfield{doi}}
\AtEveryCitekey{\clearfield{doi}}
% Month
\AtEveryBibitem{\clearfield{month}}
\AtEveryCitekey{\clearfield{month}}
% Day
\AtEveryBibitem{\clearfield{day}}
\AtEveryCitekey{\clearfield{day}}
% language
\AtEveryBibitem{\clearlist{language}}
\AtEveryCitekey{\clearlist{language}}
\AtEveryBibitem{\clearlist{langid}}
\AtEveryCitekey{\clearlist{langid}}

% Call bib file
%\addbibresource{NPS_merged.bib}
\addbibresource{NPS.bib}
\addbibresource{NPS2.bib}
\addbibresource{NPS3.bib}
% Hide certain fields from biber	
\DeclareSourcemap{
	\maps[datatype=biblatex]{
		\map{
			\step[fieldset=number, null]
			\step[fieldset=issue, null]
			\step[fieldset=note, null]
			\step[fieldset=urldate, null]
			\step[fieldset=url, null]
			\step[fieldset=langid, null]
			\step[fieldset=issn, null]
			\step[fieldset=isbn, null]
			\step[fieldset=doi, null]
			\step[fieldset=month, null]
			\step[fieldset=eprint, null]
			\pertype{article}
			\step[fieldset=month, null]
			\step[fieldset=eprinttype, null]
			\step[fieldset=issn, null]
			%\step[fieldset=editor, null]
}   }  }

% Introduce \centerfloat command to get figure wider than \textwidth
\makeatletter
\newcommand*{\centerfloat}{%
	\parindent \z@
	\leftskip \z@ \@plus 1fil \@minus \textwidth
	\rightskip\leftskip
	\parfillskip \z@skip}
\makeatother


\usepackage[autostyle=true]{csquotes} % Required to generate language-dependent quotes in the bibliography
